%latexmk -pdf -pvc Notes
\documentclass{article}
\setlength{\oddsidemargin}{6pt}
\setlength{\textwidth}{440pt}
\vspace{-8ex}
\usepackage{mathtools}
\usepackage{enumitem}
\DeclarePairedDelimiter{\ceil}{\lceil}{\rceil}
\usepackage{listings}
\usepackage{amsmath}
\usepackage{color}
\usepackage{graphicx}
\usepackage[left=0.70in, right=0.70in, top=0.50in, left=0.70in, headsep=0pt]{geometry}
\usepackage{hyperref}
\hypersetup{
    colorlinks,
    citecolor=black,
    filecolor=black,
    linkcolor=black,
    urlcolor=black
}
\graphicspath{ {./assets/} }

\definecolor{dkgreen}{rgb}{0,0.6,0}
\definecolor{gray}{rgb}{0.5,0.5,0.5}
\definecolor{mauve}{rgb}{0.58,0,0.82}

\lstset{frame=tb,
  language=Java,
  aboveskip=3mm,
  belowskip=3mm,
  showstringspaces=false,
  columns=flexible,
  basicstyle={\small\ttfamily},
  numbers=none,
  numberstyle=\tiny\color{gray},
  keywordstyle=\color{blue},
  commentstyle=\color{dkgreen},
  stringstyle=\color{mauve},
  breaklines=true,
  breakatwhitespace=true,
  tabsize=3
}
\usepackage{hyperref}
\hypersetup{
    colorlinks=true,
    linkcolor=blue,
    filecolor=magenta,
    urlcolor=cyan,
}

\title{Programming Interview Notes}
\date{}
\begin{document} 
  \author{Michael Li}
  \title{Programming Interview Notes}
  \maketitle
  \tableofcontents
  \newpage
  \section{DFS}
  \begin{lstlisting}
  stack.push(element);
  while(!stack.isEmpty()) {
    T curr = stack.pop();
    for (e : adj(curr)) {
      if(!visited[e]) {
        visited[e] = true;        //Keep track of unvisited using array
        stack.push(e); 
      }
    }
  }
  \end{lstlisting}
  For graphs, need to figure out how to use visited array to match requirements\\
  \href{https://leetcode.com/problems/number-of-connected-components-in-an-undirected-graph/}{323. Number of Connected Components in an Undirected Graph
} \quad \href{https://leetcode.com/problems/number-of-islands/}{200. Number of Islands
} \quad \href{https://leetcode.com/problems/friend-circles/description/}{547. Friend Circles
}
  \section{BFS}
  \begin{lstlisting}
  queue.add(element);
  while(!queue.isEmpty()) {
    T curr = queue.poll();
    for (e : adj(curr)) {
      if(!visited[e]) {
        visited[e] = true;      //keep track of unvisited using array
        queue.add(e);
      }
    }
  }
  \end{lstlisting}
  Key issue to recognize when to use DFS (look at deep as possible first) or BFS (process neighbors first because going deep first might be inefficient)\\
  \href{https://leetcode.com/problems/binary-tree-level-order-traversal/}{102. Binary Tree Level Order Traversal} \quad \href{https://leetcode.com/problems/01-matrix/}{542. 01 Matrix}
  \section{Tree Traversal (DFS on Trees)}
  \subsection{Preorder Traversal}
  root, Left, Right
  \begin{lstlisting}
  public List<Integer> preorderTraversal(TreeNode root) {
    List<Integer> list = new ArrayList();
    if(root == null) return list;
    Stack<TreeNode> stack = new Stack();
    stack.push(root);
    
    while(!stack.isEmpty()) {
      TreeNode curr = stack.pop();
      list.add(curr.val);
      if(curr.right != null) stack.push(curr.right);
      if(curr.left != null) stack.push(curr.left);
    }
    return list;
  }
  \end{lstlisting}
  \href{https://leetcode.com/problems/binary-tree-preorder-traversal/}{144. Binary Tree Preorder Traversal}
  \subsection{Inorder Traversal}
  Left, root, Right \\
  Useful when you need to iterate through a tree inorder or retrieve the kth element
  \begin{lstlisting}
  public List<Integer>> inorderTraversal(TreeNode root) {
    List<Integer> list = new ArrayList();
    if(root == null) return list;
    
    Stack<TreeNode> stack = new Stack();      //Use stack for DFS
    while(root != null && !stack.isEmpty()) {
      while(root != null) {                   //Find left-most element
        stack.push(root);
        root = root.left;
      }
      root = stack.pop();
      list.add(root.val);
      root = root.right;                      //Look at left-most element's right child
    }
  }
  \end{lstlisting}
  \href{https://leetcode.com/problems/binary-tree-inorder-traversal/description/}{94. Binary Tree Inorder Traversal} \quad \href{https://leetcode.com/problems/validate-binary-search-tree/}{98. Validate Binary Search Tree} \quad \href{https://leetcode.com/problems/kth-smallest-element-in-a-bst/solution/}{230. Kth Smallest Element in a BST}

  \section{Sliding Window}
  Use two pointers and keep sliding the right pointer, examining the new character and updating the count table. Once a condition (usually duplicate) is found, start removing elements from the left pointer, shortening the window.\\
  \href{https://leetcode.com/problems/minimum-window-substring/description/}{76. Minimum Window Substring}
  \begin{lstlisting}
  public String minWindow(String s, String t) {
    Map<Character, Integer> map = new HashMap();
    // populate hashmap with character frequencies
    for(char c : t.toCharArray()) {
      if(map.containsKey(c))  map.put(c, map.get(c) + 1);
      else map.put(c, 1);
    }
    int left = 0;
    int right = 0;
    int slen = s.length();
    int count = map.size();   // used to check how many characters still need to be processed
    int len = slen;           // length of answer string
    String ans = s;

    while(right < slen) {
      char rightChar = s.charAt(right);
      if(map.containsKey(rightChar)) {
        map.put(rightChar, map.get(rightChar) - 1);
        if(map.get(rightChar) == 0) count--;
      }
      right++;
      while(count == 0) {
        // normally is  right - left + 1 but right++ happened above so indexing is correct
        if(right - left < len) {
          len = right - left;
          ans = s.substring(left, right);
        }
        char leftChar = s.charAt(left);
        if(map.containsKey(leftChar)) {
          map.put(leftChar, map.get(leftChar) + 1);
          if(map.get(leftChar) > 0) count++;
        }
        left++;
      }
    }
    return ans;
  }
  \end{lstlisting}
  \href{https://leetcode.com/problems/find-all-anagrams-in-a-string/description/}{438. Find All Anagrams in a String} \quad \href{https://leetcode.com/problems/substring-with-concatenation-of-all-words/description/}{30. Substring with Concatenation of All Words} \quad \href{https://leetcode.com/problems/longest-substring-without-repeating-characters/}{3. Longest Substring Without Repeating Characters} \quad \href{https://leetcode.com/problems/longest-substring-with-at-most-two-distinct-characters/}{159. Longest Substring with At Most Two Distinct Characters} \quad \href{https://leetcode.com/problems/permutation-in-string/description/}{567. Permutation in String}
  \section{Backtracking}
  \begin{enumerate}[noitemsep]
    \item Iterate through all possible configurations of search space (permutations or subsets)
    \item Generate each configuration once by defining a generation order
    \item At each step, try to extend a partial solution by adding an element. Test if it's a solution
      \begin{itemize}[noitemsep]
        \item if yes then process it
        \item if no then recurse down and check if it is a partial solution
      \end{itemize}
    \item Pruning: cut off search if we know it is not part of the solution (e.g. cost $>$ curr min)
  \end{enumerate}
  \begin{lstlisting}
  public List<List<Integer>> subsets(int[] nums) {
      List<List<Integer>> list = new ArrayList(); //solution array
      backtrack(list, new ArrayList(), 0, nums);
      return list;
  }
  
  public void backtrack(List<List<Integer>> list, List<Integer> sublist, int k, int[] nums) {
      if (isSolution(sublist)) {
        process(sublist);                        //if sublist is a valid solution, process it
      }
      for (int i = k; i < nums.length; i++) {
          sublist.add(nums[i]);                  //extend partial solution
          backtrack(list, sublist, i + 1, nums); //recurse down
          sublist.remove(sublist.size() - 1);    //remove extension
      }
  }
  \end{lstlisting}
  \href{https://leetcode.com/problems/subsets/description/}{78. Subsets
} \quad \href{https://leetcode.com/problems/subsets-ii/description/}{90. Subsets II} \quad \href{https://leetcode.com/problems/combination-sum/description/}{39. Combination Sum} \quad \href{https://leetcode.com/problems/combination-sum-ii/description/}{40. Combination Sum II} \quad \href{https://leetcode.com/problems/n-queens/description/}{51. N-Queens}
  \section{Dynamic Programming}
  Solves by combining optimal solutions to subproblems that overlap. Divide and conquer only work for disjoint sets otherwise there's a lot of repetitive recursive calls. Howeever, dynamic programming uses additional memory to save subproblem solutions so there is a time-memory tradeoff. 
  \begin{enumerate}[noitemsep]
    \item Characterize structure of optimal solution 
    \item Recursively define value of optimal solution 
    \item Compute value of optimal solution (usually bottom-up approach)
    \item Construct optimal solution from computed info
  \end{enumerate}
  Two approaches to dynamic programming:
  \begin{itemize}[noitemsep]
    \item Top-down with memoization: write the procedure recurisvely but saves result of each subproblem in an array. The procedure will first check to see if the array has the solution to the subproblem; if not then the value is computed normally.
    \item Bottom-up: intuition is to sort by subproblem size and solve them in size order (smallest first). When solving a particular subproblem, we have already saved the solutions to the smaller subproblems.
  \end{itemize}
  Look for an optimal substructure: optimal solution requires optimal solution to subproblems
  \begin{itemize}[noitemsep]
    \item Solution consists of making a choice
    \item Assume given choice leads to optimal solutino 
    \item determine which subproblems occur
    \item show solutions to subproblems used within optimal solution must also be optimal
  \end{itemize}
  Dynamic Programming makes choice at each step based on solutions to sub problems so we do bottom-up approach, using best subproblems to solve later problems or top-down with memoization. In both cases, we always end up solving the smaller subproblems then solving bigger ones.
  \section{Greedy Algorithm}
  \textbf{Greedy Choice Property: }Idea is to make locally optimal choices in hopes that it will lead to a globally optimal solution.
  \begin{itemize}[noitemsep]
    \item Determine optimal substructure
    \item Develop recursive solution 
    \item Show if we make greedy choice, only one subproblem remains
    \item Prove it's safe to make greedy choice
    \item Develop recursive algorithm for greedy choice
    \item Convert recursive greedy to iterative
  \end{itemize}
  We want to make greedy choice then solve the subproblems that remain. This is usually a top-down approach where one greedy choice is made after another, reducing the size of the problem
  \section{Miscellaneous}
  \subsection{Java For Each Loop on Map}
  \begin{lstlisting}
  for(Map.Entry<Key, Value> e : map.entrySet()) {
    e.getKey();
    e.getValue();
  }
  \end{lstlisting}
  \subsection{Implementing Java Comparator}
  \begin{lstlisting}
  Set<String> set = new TreeSet(new Comparator<String>() {
    @Override
    public int compare(String s1, String s2) {
      return s1.compareTo(s2);
    }
  });
  \end{lstlisting}
  \subsection{Linked List Dummy Head}
  Using dummy head node avoids having to prepopulating the head with a value prior to iterating through the linked list. At the end just return dummy.next. \href{https://leetcode.com/problems/merge-two-sorted-lists/}{21. Merge Two Sorted Lists
}
  \begin{lstlisting}
  public ListNode mergeTwoLists(ListNode l1, ListNode l2) {
    ListNode dummy = new ListNode(0);
    ListNode curr = dummy;
    while(l1 != null && l2 != null) {
        if(l1.val < l2.val) {
            curr.next = new ListNode(l1.val);
            l1 = l1.next;
        } else {
            curr.next = new ListNode(l2.val);
            l2 = l2.next;
        }
        curr = curr.next;
    }
    if(l1 != null) {
        curr.next = l1;
    }
    if(l2 != null) {
        curr.next = l2;
    }
    return dummy.next;
  }
  \end{lstlisting}
  \subsection{Modified Binary Search}
  Returns the target index by finding the left neighbor's index then returning mid + 1.
  \begin{lstlisting}
  //find target by finding the element to the left of target then do index + 1.
  public int modifiedBS(int[], int l, int r, int target) {
    int left = l;
    int right = r;
    while(left <= right) {
      int mid = (left + right) / 2;
      if(nums[mid + 1] == target) {
        return mid + 1;
      }
      if(nums[mid] > target) {
        right = mid - 1;
      } else {
        left = mid + 1;
      }
    }
  }
  \end{lstlisting}
  \href{https://leetcode.com/problems/find-first-and-last-position-of-element-in-sorted-array/}{34. Find First and Last Position of Element in Sorted Array} \quad \href{https://leetcode.com/problems/search-in-rotated-sorted-array/submissions/}{33. Search in Rotated Sorted Array}
  \subsection{Trie}

\end{document}
